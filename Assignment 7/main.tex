\documentclass{article}
\usepackage[utf8]{inputenc}
\usepackage{color}
\usepackage[margin=1 in]{geometry}

\title{Assignment 7}
\author{Christian Mathieu Schmidt (2537621)
\and Simon Laurent Lebailly (2549365)\\
\and Group 2H}

\usepackage{graphicx}
\usepackage{gensymb}

%
\usepackage{amsmath, amsthm, amssymb}
\usepackage[ngerman, english]{babel}
\usepackage{marvosym}
\usepackage{graphics}
\usepackage{extarrows}
\usepackage{forloop}
\usepackage{mathtools}
\usepackage{enumitem}

\usepackage[]{algorithm2e}

\usepackage{hyperref}% http://ctan.org/pkg/hyperref
\usepackage{cleveref}% http://ctan.org/pkg/cleveref
\usepackage{lipsum}% http://ctan.org/pkg/lipsum
\newtheorem{definition}{Definition}
\newtheorem{theorem}{Theorem}
\newtheorem{lemma}{Lemma}
\newtheorem{preliminary}{Preliminary}
\newtheorem{notation}{Notation}
\newtheorem{property}{Property}
\newtheorem{corollary}{Corollary}
\newtheorem{example}{Example}
\newtheorem{hypothesis}{Hypothesis}

\crefname{theorem}{Theorem}{Theorems}
\crefname{definition}{Definition}{Definitions}
\crefname{lemma}{Lemma}{Lemmas}
\crefname{preliminary}{Preliminary}{Preliminaries}
\crefname{notation}{Notation}{Notations}
\crefname{property}{Property}{Properties}
\crefname{corollary}{Corollary}{Corollaries}
\crefname{example}{Example}{Examples}
\crefname{hypothesis}{Hypothesis}{Hypotheses}

\newenvironment{beweis}{\begin{proof}[Beweis]}{\end{proof}}
%
 

\begin{document}

\maketitle


\section*{7.1 Color Space} \label{ex1}
    \begin{itemize}
        \item CIE-XYZ transformation:\\
        We need to find the Transformation $M$ such that:
        $$M \left(\begin{matrix} R\\G\\B \end{matrix}\right)  = \left( \begin{matrix} X_r & X_g & X_b \\ Y_r & Y_g & Y_b \\ Z_r & Z_g & Z_b \end{matrix} \right)\left(\begin{matrix} R\\G\\B \end{matrix}\right) = \left( \begin{matrix} x_rC_r & x_gC_g & x_bC_b \\ y_rC_r & y_gC_g & y_bC_b \\ z_rC_r & z_gC_g & z_bC_b \end{matrix} \right)\left(\begin{matrix} R\\G\\B \end{matrix}\right) = \left(\begin{matrix} X\\Y\\Z \end{matrix}\right)$$
        Based on the specification of default CIE-XYZ values for monitors, we can complete the matrix as follows:
        $$M  = \left( \begin{matrix} 0.64 * C_r & 0.3 * C_g & 0.15 * C_b \\ 0.33 * C_r & 0.6 * C_g & 0.06 * C_b \\ (1-0.64-0.33)C_r & (1-0.3-0.6)C_g & (1-0.15-0.06)C_b \end{matrix} \right) = \left( \begin{matrix} 0.64 * C_r & 0.3 * C_g & 0.15 * C_b \\ 0.33 * C_r & 0.6 * C_g & 0.06 * C_b \\ 0.03 * C_r & 0.1 * C_g & 0.79 * C_b \end{matrix} \right)$$
        Furthermore, we also know the white values $x_w = 0.3127$ and $y_w = 0.3290$, and the RGB Values of the color white is (1, 1, 1):
        $$M \left(\begin{matrix} 1\\1\\1 \end{matrix}\right)  = \left( \begin{matrix} 0.64 * C_r & 0.3 * C_g & 0.15 * C_b \\ 0.33 * C_r & 0.6 * C_g & 0.06 * C_b \\ 0.03 * C_r & 0.1 * C_g & 0.79 * C_b \end{matrix} \right)\left(\begin{matrix} 1\\1\\1 \end{matrix}\right) = \left(\begin{matrix} X_w \\ Y_w \\ Z_w\end{matrix}\right)$$
        If we normalize the constant $Y_w = 1$, we then get the expression:
        $$Y_w = y_wC_w \Leftrightarrow 1 = 0.3290 * C_w \Leftrightarrow C_w = \frac{1}{0.3290}$$
        Based on $C_w$, we can deduce the other values of $X_w$ and $Z_w$:
        \begin{align}
            &X_w = x_wC_w = 0.3127 * \frac{1}{0.3290} = 0.9505\\
            &Y_w = 1\\
            &Z_w = x_wC_w = (1-0.3127 - 0.3290) * \frac{1}{0.3290} = \frac{0.3583}{0.3290} = 1.0891
        \end{align}
        If we now solve the linear equation, we get out transformation matrix as follows:
        $$M = \left( \begin{matrix} 0.64 * 0.6463 & 0.3 * 1.1921 & 0.15 * 1.1917 \\ 0.33 * 0.6463 & 0.6 * 1.1921 & 0.06 * 1.1917 \\ 0.03 * 0.6463 & 0.1 * 1.1921 & 0.79 * 1.1917 \end{matrix} \right) = \left( \begin{matrix} 0.413632 & 0.35763 & 0.178755 \\ 0.213279 & 0.71526 & 0.071502 \\ 0.019389 & 0.11921 & 0.941443 \end{matrix} \right)$$
        We can now use this matrix to compute our XYZ values for the RGB Color (1, 0, 1):
        $$M \left(\begin{matrix} 1 \\ 0 \\ 1\end{matrix}\right) = \left( \begin{matrix} 0.413632 & 0.35763 & 0.178755 \\ 0.213279 & 0.71526 & 0.071502 \\ 0.019389 & 0.11921 & 0.941443 \end{matrix} \right) \left(\begin{matrix} 1 \\ 0 \\ 1\end{matrix}\right) = \left(\begin{matrix} 0.413632 + 0.178755 \\ 0.213279 + 0.071502 \\ 0.019389 + 0.941443\end{matrix}\right) = \left(\begin{matrix} 0.5924 \\ 0.2848 \\ 0.9608\end{matrix}\right)$$
        
        \item The gamma correction is irrelevant for this particular color as all RGB Values of this color are at the extreme ends of the spectrum, meaning a correction is not necessary.
    \end{itemize}
    
    
    
    

\end{document}

