\documentclass{article}
\usepackage[utf8]{inputenc}
\usepackage{color}
\usepackage[margin=1 in]{geometry}

\title{Assignment 8}
\author{Christian Mathieu Schmidt (2537621)
\and Simon Laurent Lebailly (2549365)\\
\and Group 2H}

\usepackage{graphicx}
\usepackage{gensymb}

%
\usepackage{amsmath, amsthm, amssymb}
\usepackage[ngerman, english]{babel}
\usepackage{marvosym}
\usepackage{graphics}
\usepackage{extarrows}
\usepackage{forloop}
\usepackage{mathtools}
\usepackage{enumitem}
\usepackage{tikz}

\usepackage[]{algorithm2e}

\usepackage{hyperref}% http://ctan.org/pkg/hyperref
\usepackage{cleveref}% http://ctan.org/pkg/cleveref
\usepackage{lipsum}% http://ctan.org/pkg/lipsum
\newtheorem{definition}{Definition}
\newtheorem{theorem}{Theorem}
\newtheorem{lemma}{Lemma}
\newtheorem{preliminary}{Preliminary}
\newtheorem{notation}{Notation}
\newtheorem{property}{Property}
\newtheorem{corollary}{Corollary}
\newtheorem{example}{Example}
\newtheorem{hypothesis}{Hypothesis}

\crefname{theorem}{Theorem}{Theorems}
\crefname{definition}{Definition}{Definitions}
\crefname{lemma}{Lemma}{Lemmas}
\crefname{preliminary}{Preliminary}{Preliminaries}
\crefname{notation}{Notation}{Notations}
\crefname{property}{Property}{Properties}
\crefname{corollary}{Corollary}{Corollaries}
\crefname{example}{Example}{Examples}
\crefname{hypothesis}{Hypothesis}{Hypotheses}

\newenvironment{beweis}{\begin{proof}[Beweis]}{\end{proof}}
%
 

\begin{document}

\maketitle


\section*{8.1 Clipping (15 Points)} \label{ex1}

Using the algorithm presented on the slides, we get the following computation:
\begin{itemize}
    \item Line $P_4 - P_1$:\\
    OC($P_4$) = 0110, OC($P_1$) = 1000, edge = 0
    \begin{itemize}
        \item OC($P_4$) AND OC($P_1$) = 0000 = 0
        \item OC($P_4$) OR OC($P_1$) = 1110 $\neq$ 0
        \item (OC($P_4$) XOR OC($P_1$))[0] = 1110[0] = 1
            \begin{itemize}
                \item[] $\Rightarrow$ Clipping on this edge:
                \item[] $P_4$[0] = 0
                \item[] $P_1$[0] = 1: $P_1 \Rightarrow P_1'$
            \end{itemize}
    \end{itemize}
    
    \item Line $P_1 - P_2$:\\
    OC($P_1'$) = 0000, OC($P_2$) = 0001, edge = 0
    \begin{itemize}
        \item OC($P_1$) AND OC($P_2$) = 0000 = 0
        \item OC($P_1$) OR OC($P_2$) = 0001 $\neq$ 0
        \item (OC($P_1$) XOR OC($P_2$))[0] = 0001[0] = 0
        \begin{itemize}
            \item[] $\Rightarrow$ no clipping with regards to this edge
        \end{itemize}
    \end{itemize}
    
    \item Line $P_2 - P_3$:\\
    OC($P_2$) = 0001, OC($P_3$) = 1010, edge = 0
    \begin{itemize}
        \item OC($P_2$) AND OC($P_3$) = 0000 = 0
        \item OC($P_2$) OR OC($P_3$) = 1011 $\neq$ 0
        \item (OC($P_2$) XOR OC($P_3$))[0] = 1011[0] = 1
            \begin{itemize}
                \item[] $\Rightarrow$ Clipping on this edge:
                \item[] $P_2$[0] = 0
                \item[] $P_3$[0] = 1: $P_3 \Rightarrow P_3'$
            \end{itemize}
    \end{itemize}
    
    \item Line $P_3 - P_4$:\\
    OC($P_3$) = 0001, OC($P_4$) = 1010, edge = 0
    \begin{itemize}
        \item OC($P_2$) AND OC($P_3$) = 0000 = 0
        \item OC($P_2$) OR OC($P_3$) = 1011 $\neq$ 0
        \item (OC($P_2$) XOR OC($P_3$))[0] = 1011[0] = 1
        \begin{itemize}
            \item[] $\Rightarrow$ no clipping with regards to this edge
        \end{itemize}
    \end{itemize}
    
\end{itemize}

\newpage
	\begin{center}
		\begin{tikzpicture}[scale=0.75]
			%Zeichne Gitter der Matrix
				\draw[gray] (0,0) -- (0,6);
				\draw[gray] (0,6) -- (12,6);
				\draw[gray] (12,6) -- (12,0);
				\draw[gray] (0,0) -- (12,0);
				
			%User vector
				\node at (-0.5,-0.5) {$C_1$};
				\node at (-0.5,6.5) {$C_2$};
				\node at (12.5,6.5) {$C_3$};
				\node at (12.5,-0.5) {$C_4$};
				
				\node at (0,0) {$\times$};
				\node at (0,6) {$\times$};
				\node at (12,6) {$\times$};
				\node at (12,0) {$\times$};
				
				
				\node at (9,8.5) {$P_1$};
				\node at (-6.5,3) {$P_2$};
				\node at (14,20.5) {$P_3$};
				\node at (15.5,-4.5) {$P_4$};
				
				\node at (9,8) {$\times$};
				\node at (-6,3) {$\times$};
				\node at (14,20) {$\times$};
				\node at (15,-4) {$\times$};
				

				\draw[black] (9,8) -- (-6,3);
				\draw[black] (-6,3) -- (14,20);
				\draw[black] (14,20) -- (15,-4);
				\draw[black] (15,-4) -- (9,8);
				
				
				\node[red,thick] at (0.5,4.5) {$P_1''$};
				\node[red,thick] at (3.5,5.5) {$P_1'$};
				\node[red,thick] at (9.5,5.5) {$P_4''$};
				\node[red,thick] at (11.5,1.5) {$P_4'$};
				
				\node[red,thick] at (0,5) {$\times$};
				\node[red,thick] at (3,6) {$\times$};
				\node[red,thick] at (10,6) {$\times$};
				\node[red,thick] at (12,2) {$\times$};
				
				\draw[red, thick] (0,5) -- (3,6) -- (0,6) -- (0,5);
				\draw[red, thick] (10,6) -- (12,6) -- (12,2) -- (10,6);

				
			%Item vector ^T
				%draw[step=1cm,gray,very thin] (9,1) grid (14,2);	

				
  		\end{tikzpicture}
  	\end{center}


\newpage
\section*{8.2 Hermite Spline (15 Points)} \label{ex2}
    We have the following cubic polynomial:
        \begin{align}
            p(t) &= at^3 + bt^2 + ct + d
        \end{align}
    with $a,b,c,d \in \mathbb{R}^3$
    
    \subsection*{a)}
        We have the following constraints to $p$
        \begin{align}
            p(0) &= \left(\begin{matrix} a_1\\a_2\\a_3 \end{matrix}\right) \cdot 0^3 + \left(\begin{matrix} b_1\\b_2\\b_3 \end{matrix}\right) \cdot 0^2
                + \left(\begin{matrix} c_1\\c_2\\c_3 \end{matrix}\right) \cdot 0 + \left(\begin{matrix} d_1\\d_2\\d_3 \end{matrix}\right)
                = \left(\begin{matrix} d_1\\d_2\\d_3 \end{matrix}\right)
                = \left(\begin{matrix} 0\\1\\0 \end{matrix}\right)\\
            p(1) &= \left(\begin{matrix} a_1\\a_2\\a_3 \end{matrix}\right) \cdot 1^3 + \left(\begin{matrix} b_1\\b_2\\b_3 \end{matrix}\right) \cdot 1^2
                + \left(\begin{matrix} c_1\\c_2\\c_3 \end{matrix}\right) \cdot 1 + \left(\begin{matrix} d_1\\d_2\\d_3 \end{matrix}\right)
                = \left(\begin{matrix} a_1\\a_2\\a_3 \end{matrix}\right) + \left(\begin{matrix} b_1\\b_2\\b_3 \end{matrix}\right)
                + \left(\begin{matrix} c_1\\c_2\\c_3 \end{matrix}\right) + \left(\begin{matrix} d_1\\d_2\\d_3 \end{matrix}\right)
                = \left(\begin{matrix} 1\\0\\1 \end{matrix}\right)
        \end{align}
        Now we will compute the first derivative $p'$ of $p$ with respect to $t$
        \begin{align}
            p'(t) &= \frac{\partial}{\partial t} p = 3at^2 + 2bt + c
        \end{align}
        and have two further constraints
        \begin{align}
            p'(0) &= 3 \left(\begin{matrix} a_1\\a_2\\a_3 \end{matrix}\right) \cdot 0^2 + 2 \left(\begin{matrix} b_1\\b_2\\b_3 \end{matrix}\right) \cdot 0
                + \left(\begin{matrix} c_1\\c_2\\c_3 \end{matrix}\right) = \left(\begin{matrix} c_1\\c_2\\c_3 \end{matrix}\right)
                = \left(\begin{matrix} 1\\0\\1 \end{matrix}\right)\\
            p'(1) &= 3 \left(\begin{matrix} a_1\\a_2\\a_3 \end{matrix}\right) \cdot 1^2 + 2 \left(\begin{matrix} b_1\\b_2\\b_3 \end{matrix}\right) \cdot 1
                + \left(\begin{matrix} c_1\\c_2\\c_3 \end{matrix}\right) 
                = 3 \left(\begin{matrix} a_1\\a_2\\a_3 \end{matrix}\right) + 2 \left(\begin{matrix} b_1\\b_2\\b_3 \end{matrix}\right)
                + \left(\begin{matrix} c_1\\c_2\\c_3 \end{matrix}\right) = \left(\begin{matrix} 0\\-1\\0 \end{matrix}\right)
        \end{align}
        From (2) and (5) we can directly follow that
        \begin{align}
            \left(\begin{matrix} c_1\\c_2\\c_3 \end{matrix}\right) = \left(\begin{matrix} 1\\0\\1 \end{matrix}\right)\ \ \ \ \ \ \ \ \ \ 
            &\text{and}\ \ \ \ \ \ \ \ \ \ 
            \left(\begin{matrix} d_1\\d_2\\d_3 \end{matrix}\right) = \left(\begin{matrix} 0\\1\\0 \end{matrix}\right)
        \end{align}
        Thus, we can reduce our problem to the following equations:
        \begin{align}
            \left(\begin{matrix} a_1\\a_2\\a_3 \end{matrix}\right) + \left(\begin{matrix} b_1\\b_2\\b_3 \end{matrix}\right) 
                + \left(\begin{matrix} 0\\1\\0 \end{matrix}\right) + \left(\begin{matrix} 1\\0\\1 \end{matrix}\right)
                = \left(\begin{matrix} 1\\0\\1 \end{matrix}\right)\ \ \ \ \ 
            &\Leftrightarrow\ \ \ \ \  \left(\begin{matrix} a_1\\a_2\\a_3 \end{matrix}\right) + \left(\begin{matrix} b_1\\b_2\\b_3 \end{matrix}\right)
                = \left(\begin{matrix} 0\\-1\\0 \end{matrix}\right)\\
            3 \left(\begin{matrix} a_1\\a_2\\a_3 \end{matrix}\right) + 2 \left(\begin{matrix} b_1\\b_2\\b_3 \end{matrix}\right)
                + \left(\begin{matrix} 0\\1\\0 \end{matrix}\right) + \left(\begin{matrix} 1\\0\\1 \end{matrix}\right)
                = \left(\begin{matrix} 0\\-1\\0 \end{matrix}\right)\ \ \ \ \ 
            &\Leftrightarrow\ \ \ \ \  3 \left(\begin{matrix} a_1\\a_2\\a_3 \end{matrix}\right) + 2 \left(\begin{matrix} b_1\\b_2\\b_3 \end{matrix}\right)
                = \left(\begin{matrix} -1\\-2\\-1 \end{matrix}\right)
        \end{align}
\newpage
        Now we have three linear equation systems to solve:
        \begin{enumerate}
            \item From $p(1)$ and $p'(1)$ we compare the first coordinates:
                \begin{align}
                    \left(\begin{matrix} 1&1\\3&2 \end{matrix}\right) \cdot \left(\begin{matrix} a_1\\b_1 \end{matrix}\right) = \left(\begin{matrix} 0\\-1 \end{matrix}\right)
                    \ \ \ \Rightarrow\ \ \  \left(\begin{array}{cc} 1&1\\3&2 \end{array} \right | \left. \begin{array}{c} 0\\-1 \end{array}\right)
                    \ \ \ \rightarrow\ \ \  \left(\begin{array}{cc} 1&0\\0&1 \end{array} \right | \left. \begin{array}{c} -1\\1 \end{array}\right)
                \end{align}
                $\Rightarrow$ $a_1 = -1$ and $b_1 = 1$.
            \item From $p(1)$ and $p'(1)$ we compare the second coordinates:
                \begin{align}
                    \left(\begin{matrix} 1&1\\3&2 \end{matrix}\right) \cdot \left(\begin{matrix} a_2\\b_2 \end{matrix}\right) = \left(\begin{matrix} -1\\-1 \end{matrix}\right)
                    \ \ \ \Rightarrow\ \ \  \left(\begin{array}{cc} 1&1\\3&2 \end{array} \right | \left. \begin{array}{c} -1\\-1 \end{array}\right)
                    \ \ \ \rightarrow\ \ \  \left(\begin{array}{cc} 1&0\\0&1 \end{array} \right | \left. \begin{array}{c} 1\\-2 \end{array}\right)
                \end{align}
                $\Rightarrow$ $a_2 = 1$ and $b_2 = -2$.
            \item From $p(1)$ and $p'(1)$ we compare the third coordinates:
                \begin{align}
                    \left(\begin{matrix} 1&1\\3&2 \end{matrix}\right) \cdot \left(\begin{matrix} a_3\\b_3 \end{matrix}\right) = \left(\begin{matrix} 0\\-1 \end{matrix}\right)
                    \ \ \ \Rightarrow\ \ \  \left(\begin{array}{cc} 1&1\\3&2 \end{array} \right | \left. \begin{array}{c} 0\\-1 \end{array}\right)
                    \ \ \ \rightarrow\ \ \  \left(\begin{array}{cc} 1&0\\0&1 \end{array} \right | \left. \begin{array}{c} -1\\1 \end{array}\right)
                \end{align}
                $\Rightarrow$ $a_3 = -1$ and $b_3 = 1$.
        \end{enumerate}
        The equation is satisfied by the following values
        \begin{align}
            \left(\begin{matrix} a_1\\a_2\\a_3 \end{matrix}\right) = \left(\begin{matrix} -1\\1\\-1 \end{matrix}\right)
            \ \ \ \ \ \ \ \ \ \ &\text{and}\ \ \ \ \ \ \ \ \ \ 
            \left(\begin{matrix} b_1\\b_2\\b_3 \end{matrix}\right) = \left(\begin{matrix} 1\\-2\\1 \end{matrix}\right)
        \end{align}
        If is obviously that $a,b,c,d$ in $p(0)$ and $p'(0)$ leads to the correct result.
        That's why we put them in $p(1)$ and $p'(1)$ to verify the result:
        \begin{align}
            p(1) &= \left(\begin{matrix} -1\\1\\-1 \end{matrix}\right) \cdot 1^3 + \left(\begin{matrix} 1\\-2\\1 \end{matrix}\right) \cdot 1^2
                + \left(\begin{matrix} 1\\0\\1 \end{matrix}\right) \cdot 1 + \left(\begin{matrix} 0\\1\\0 \end{matrix}\right)
                = \left(\begin{matrix} -1\\1\\-1 \end{matrix}\right) + \left(\begin{matrix} 1\\-2\\1 \end{matrix}\right)
                + \left(\begin{matrix} 1\\0\\1 \end{matrix}\right) + \left(\begin{matrix} 0\\1\\0 \end{matrix}\right)
                = \left(\begin{matrix} 1\\0\\1 \end{matrix}\right)\\
            p'(1) &= 3 \left(\begin{matrix} -1\\1\\-1 \end{matrix}\right) \cdot 1^2 + 2 \left(\begin{matrix} 1\\-2\\1 \end{matrix}\right) \cdot 1
                + \left(\begin{matrix} 1\\0\\1 \end{matrix}\right)
                = \left(\begin{matrix} -3\\3\\-3 \end{matrix}\right) + \left(\begin{matrix} 2\\-4\\2 \end{matrix}\right)
                + \left(\begin{matrix} 1\\0\\1 \end{matrix}\right) = \left(\begin{matrix} 0\\-1\\0 \end{matrix}\right)
        \end{align}
        So we can conclude that the searched coefficients are
        \begin{align}
            a = \left(\begin{matrix} a_1\\a_2\\a_3 \end{matrix}\right) = \left(\begin{matrix} -1\\1\\-1 \end{matrix}\right)
            \ \ \ \wedge\ \ \ 
            b = \left(\begin{matrix} b_1\\b_2\\b_3 \end{matrix}\right) = \left(\begin{matrix} 1\\-2\\1 \end{matrix}\right)
            \ \ \ &\wedge\ \ \ 
            c = \left(\begin{matrix} c_1\\c_2\\c_3 \end{matrix}\right) = \left(\begin{matrix} 1\\0\\1 \end{matrix}\right)
            \ \ \ \wedge\ \ \ 
            d = \left(\begin{matrix} d_1\\d_2\\d_3 \end{matrix}\right) = \left(\begin{matrix} 0\\1\\0 \end{matrix}\right)
        \end{align}
        

\newpage
    \subsection*{b)}
        From the slides we know that
        \begin{align}
            p(t) &= (T \cdot M \cdot G)^T
        \end{align}
        with
        \begin{align}
            & &T &:= \left(\begin{matrix} t^3 & t^2 & t & 1 \end{matrix}\right)\\
            &\wedge
            &M &:= \left(\begin{matrix} 0&0&0&1 \\ 1&1&1&1 \\0&0&1&0 \\ 1&0&0&0 \end{matrix}\right)^{-1} 
                = \left(\begin{matrix} 2&-2&1&1 \\ -3&3&-2&-1 \\ 0&0&1&0 \\ 1&0&0&0 \end{matrix}\right)\\
            &\wedge 
            &G &:= \left(\begin{matrix} p(0)^T \\ p(1)^T \\ p'(0)^T \\ p'(1)^T \end{matrix}\right)
                = \left(\begin{matrix} 0&1&0 \\ 1&0&1 \\ 1&0&1 \\ 0&-1&0 \end{matrix}\right)
        \end{align}
        We first compute $p(t)^T$:
        \begin{align}
            p(t)^T = T \cdot M \cdot G
            &= \left(\begin{matrix} t^3 & t^2 & t & 1 \end{matrix}\right)
                \cdot \left(\begin{matrix} 2&-2&1&1 \\ -3&3&-2&-1 \\ 0&0&1&0 \\ 1&0&0&0 \end{matrix}\right)
                \cdot \left(\begin{matrix} 0&1&0 \\ 1&0&1 \\ 1&0&1 \\ 0&-1&0 \end{matrix}\right)\\
            &= \left(\begin{matrix} 2t^3-3t^2+1 \\ -2t^3+3t^2 \\ t^3-2t^2+t \\ t^3-t^2 \end{matrix}\right)^T
                \cdot \left(\begin{matrix} 0&1&0 \\ 1&0&1 \\ 1&0&1 \\ 0&-1&0 \end{matrix}\right)\\
            &= \left(\begin{matrix} (-2t^3+3t^2) + (t^3-2t^2+t) \\ (2t^3-3t^2+1) - (t^3-t^2) \\ (-2t^3+3t^2) + (t^3-2t^2+t) \end{matrix}\right)^T
            = \left(\begin{matrix} -t^3+t^2+t \\ t^3-2t^2+1 \\ -t^3+t^2+t \end{matrix}\right)^T
        \end{align}
        Thus we can follow
        \begin{align}
            p(t) &= \left(\begin{matrix} -1\\1\\-1 \end{matrix}\right) \cdot t^3 + \left(\begin{matrix} 1\\-2\\1 \end{matrix}\right) \cdot t^2
                + \left(\begin{matrix} 1\\0\\1 \end{matrix}\right) \cdot t + \left(\begin{matrix} 0\\1\\0 \end{matrix}\right)
        \end{align}
        Now we can conclude that the searched coefficients are
        \begin{align}
            a = \left(\begin{matrix} a_1\\a_2\\a_3 \end{matrix}\right) = \left(\begin{matrix} -1\\1\\-1 \end{matrix}\right)
            \ \ \ \wedge\ \ \ 
            b = \left(\begin{matrix} b_1\\b_2\\b_3 \end{matrix}\right) = \left(\begin{matrix} 1\\-2\\1 \end{matrix}\right)
            \ \ \ &\wedge\ \ \ 
            c = \left(\begin{matrix} c_1\\c_2\\c_3 \end{matrix}\right) = \left(\begin{matrix} 1\\0\\1 \end{matrix}\right)
            \ \ \ \wedge\ \ \ 
            d = \left(\begin{matrix} d_1\\d_2\\d_3 \end{matrix}\right) = \left(\begin{matrix} 0\\1\\0 \end{matrix}\right)
        \end{align}
    
    
    

\end{document}

