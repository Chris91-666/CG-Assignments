\documentclass{article}
\usepackage[utf8]{inputenc}
\usepackage{color}
\usepackage[margin=1 in]{geometry}

\title{Assignment 3}
\author{Christian Mathieu Schmidt (2537621)
\and Simon Laurent Lebailly (2549365)\\
\and Group 2H}

\usepackage{graphicx}
\usepackage{gensymb}

%
\usepackage{amsmath, amsthm, amssymb}
\usepackage[ngerman, english]{babel}
\usepackage{marvosym}
\usepackage{graphics}
\usepackage{extarrows}
\usepackage{forloop}
\usepackage{mathtools}

\usepackage[]{algorithm2e}

\usepackage{hyperref}% http://ctan.org/pkg/hyperref
\usepackage{cleveref}% http://ctan.org/pkg/cleveref
\usepackage{lipsum}% http://ctan.org/pkg/lipsum
\newtheorem{definition}{Definition}
\newtheorem{theorem}{Theorem}
\newtheorem{lemma}{Lemma}
\newtheorem{preliminary}{Preliminary}
\newtheorem{notation}{Notation}
\newtheorem{property}{Property}
\newtheorem{corollary}{Corollary}
\newtheorem{example}{Example}
\newtheorem{hypothesis}{Hypothesis}

\crefname{theorem}{Theorem}{Theorems}
\crefname{definition}{Definition}{Definitions}
\crefname{lemma}{Lemma}{Lemmas}
\crefname{preliminary}{Preliminary}{Preliminaries}
\crefname{notation}{Notation}{Notations}
\crefname{property}{Property}{Properties}
\crefname{corollary}{Corollary}{Corollaries}
\crefname{example}{Example}{Examples}
\crefname{hypothesis}{Hypothesis}{Hypotheses}

\newenvironment{beweis}{\begin{proof}[Beweis]}{\end{proof}}
%
 

\begin{document}

\maketitle


\section*{3.1 Node count in a tree} \label{ex1}
    Let $I$ be the number of internal nodes and $L$ be the number of leaf nodes. So it holds
    $$L = (A-1) \cdot I + 1 (I)$$
    The parent of a parent node has at most $A-1$ leafs, and at least one child.
    Parents witch have only leafs, have exactly $A$ leafs.
    Every time a leaf switches to a parent, we loose a leaf, add one internal node, and become $A$ new leafs.
    In sum, the number of internal nodes increases by one, and the number of leafs increases by $A-1$.
    And in conclusion we become (I).
    Now we can conclude, that we have $(A-1) \cdot I$ leafs, but we have also to add the root node by addition $+1$.\\
    (The more mathematical kind of proof would be via structural induction, but the description of the exercise is only "Derive a relation between the number of internal nodes I and leaf nodes L.", and not "prove your relation")

\section*{3.2 Planar Surface Area Heuristic} \label{ex2}
    


\end{document}