\documentclass{article}
\usepackage[utf8]{inputenc}
\usepackage{color}
\usepackage[margin=1 in]{geometry}

\title{Assignment 6}
\author{Christian Mathieu Schmidt (2537621)
\and Simon Laurent Lebailly (2549365)\\
\and Group 2H}

\usepackage{graphicx}
\usepackage{gensymb}

%
\usepackage{amsmath, amsthm, amssymb}
\usepackage[ngerman, english]{babel}
\usepackage{marvosym}
\usepackage{graphics}
\usepackage{extarrows}
\usepackage{forloop}
\usepackage{mathtools}
\usepackage{enumitem}

\usepackage[]{algorithm2e}

\usepackage{hyperref}% http://ctan.org/pkg/hyperref
\usepackage{cleveref}% http://ctan.org/pkg/cleveref
\usepackage{lipsum}% http://ctan.org/pkg/lipsum
\newtheorem{definition}{Definition}
\newtheorem{theorem}{Theorem}
\newtheorem{lemma}{Lemma}
\newtheorem{preliminary}{Preliminary}
\newtheorem{notation}{Notation}
\newtheorem{property}{Property}
\newtheorem{corollary}{Corollary}
\newtheorem{example}{Example}
\newtheorem{hypothesis}{Hypothesis}

\crefname{theorem}{Theorem}{Theorems}
\crefname{definition}{Definition}{Definitions}
\crefname{lemma}{Lemma}{Lemmas}
\crefname{preliminary}{Preliminary}{Preliminaries}
\crefname{notation}{Notation}{Notations}
\crefname{property}{Property}{Properties}
\crefname{corollary}{Corollary}{Corollaries}
\crefname{example}{Example}{Examples}
\crefname{hypothesis}{Hypothesis}{Hypotheses}

\newenvironment{beweis}{\begin{proof}[Beweis]}{\end{proof}}
%
 

\begin{document}

\maketitle


\section*{6.2 Fourier Transformation} \label{ex2}
    \begin{align}
        \mathcal{F}(B) = F : F(k) &= \int_{-\infty}^{+\infty} B(x) \cdot e^{-2\pi ikx}dx\\
        &= \int_{-R}^{R} 1 \cdot e^{-2\pi ikx}dx\\
        &= \left[ \frac{-1}{i2\pi k} \cdot e^{-2\pi ikx} \right]_{-R}^R\\
        &= \frac{-1}{i2\pi k} \cdot \left( e^{-2\pi ikR} - e^{2\pi ikR} \right)\\
        &= \frac{1}{i2\pi k} \cdot \left( e^{2\pi ikR} - e^{-2\pi ikR} \right)\\
        &= \frac{1}{i2\pi k} \cdot 2i \cdot sin(2\pi kR)\\
        &= \frac{1}{\pi k} \cdot sin(2\pi kR)
    \end{align}
    Due to the underlying restrictions we set $R=1$:
    \begin{align}
        \frac{1}{\pi k} \cdot sin(2\pi kR) &= \frac{1}{\pi k} \cdot sin(2\pi k)\\
        &= \frac{2}{2 \pi k} \cdot sin(2\pi k)\\
        &= 2 \cdot \frac{sin(2\pi k)}{2 \pi k} = 2 \cdot sinc(2 \pi k) 
    \end{align}
    
    
    
    

\end{document}

